\documentclass[runningheads,a4paper]{llncs}

\usepackage[latin1]{inputenc}
\usepackage{amssymb}
\setcounter{tocdepth}{3}
\usepackage{graphicx}
\usepackage{subfigure}

\usepackage{url}
\newcommand{\keywords}[1]{\par\addvspace\baselineskip
\noindent\keywordname\enspace\ignorespaces#1}

\begin{document}

\mainmatter  

\title{The importance of diversity in distributed and asynchronous
  evolutionary algorithms}  

\titlerunning{The importante of diversity}

\author{A. N. Onymouse%
\thanks{Anonymous Institute}}
%

%\author{R.~H. Garc\'ia Ortega\inst{1,2} \and P. Garc\'ia-S\'anchez\inst{1} \and J.~J. Merelo\inst{1} \and M.~G. Arenas\inst{1} \and  P.~A.~ Castillo\inst{1} \and A.~M. Mora\inst{1}}

%\authorrunning{R.~H. Garc\'ia Ortega et al.}

%\institute{Dept. of Computer Architecture and Technology, University of Granada, Spain
%\and Fundaci\'on I+D del Software Libre - Fidesol, Granada, Spain} % ¿Seguro que quieres dejar esto? - JJ

\maketitle

\begin{abstract}

Leveraging the power of all devices connected to the Internet for
distributed evolutionary algorithms requires using ubiquituous
platforms such as JavaScript on the browser. However, these platform
are heterogenous and ephemeral, which beg for a change in the canonical
evolutionary algorithm to architectures based in concepts such as a
population pool. In this paper we will test NodIO in a distributed
computing setup; first we will establish how the base pool-based algorithm itself
behaves and how it scales for different nodes; then we will test in in
a real volunteer computing environment in different circumstances to
assess what is the configuration that is able to behave the best from
the point of view of the performance. 

\keywords{Distributed evolutionary algorithms, internet computing,
  volunteer computing, pool-based evolutionary algorithms}
\end{abstract}

\section{Introduction}

The rest of the paper is organized as follows. Coming up next, we
present the state of the art in parameter setting of simulated
worlds. The next section \ref{sec:met} will present the methodology,
followed by the experimental setup in Section \ref{sec:exp}, whose results will be shown in Section \ref{sec:res}. Finally, we will
present our conclusions to finish the paper.

%%%%%%%%%%%%%%%%%%%%%%%%%%%%%%%%%%%%%%%%%%%%%%%%%%%%%%%%%%%%%%%%%%%%%%%%%%%%%%%
%%%%%%%%%%%%%%%%%%%%%%%%%%%%%%%%%%%%%%%%%%%%%%%%%%%%%%%%%%%%%%%%%%%%%%%%%%%%%%%

\section{State of the Art}
\label{sec:sota}

In the last few years, the scenario of distributed computing has
evolved from rigid, synchronous and static setups to fluid, asynchronous and dynamic ones, as
shown in \cite{gong2015distributed}. Besides the well known and tested
master-slave and island models, new models such as the pool-based
evolutionary algorithms
\cite{roy2009distributed,bollini1999distributed,sofea:evopar2012,sofea:naco}
have arisen. A pool-based evolutionary algorithms decouple population
from the algorithm by creating a persistent pool, that can be any kind
of server or database where individuals in different states, evaluated
or not, can be accessed. Unlike island or master-slave architectures,
the pool-based architecture lends itself to all kinds of interaction
with the algorithm, storing the whole population in the pool that is
fetched by clients and returned with some operations like fitness
calculation or a single generation performed
\cite{nogueras2015self,garcia2014unreliable} or storing and fetching some
individuals selected from the clients or the server
\cite{DBLP:conf/3pgcic/GuervosMFEL12}. Any and all modes of
interaction with the pool can be mixed, creating a configuration space
that is difficult to explore.

However, the pool based architecture with its topology-free and
asynchronous nature does have the advantage of
allowing spontaneous incorporation by just interacting with the pool
and thus been used in volunteer computing experiments
\cite{daniel:euromicro09,DBLP:journals/corr/abs-0801-1210,DBLP:journals/gpem/LaredoBGVAGF14}. As
this last paper shows, designing robust algorithms is an issue, but
predicting the performance is also a big challenge.

This performance has two components. The first is the algorithmic
performance: how many fitness evaluations are needed to reach the
solution. Since the scenario is dynamic and asynchronous it is first
complicated to design an experiment that covers a wide range of
possible situations, although some attempts have been made changing
the population structure \cite{DBLP:conf/lion/LaredoGFMACG11},
migration schemes \cite{hijaze2014investigating} or sampling
strategies \cite{nogueras2015self}. 

The second component is the scaling capabilities and the actual number
of nodes gathered for a single experiment. This component of the
performance is heavily influenced by the social network the experiment
is embedded in, and heavily related to the attitude of the user
towards lending time to a web page that is running an experiment. Some
advances on the statistical distribution of users and how this is
influenced by several factors has been presented so far
DBLP:journals/corr/GuervosG15, however, this facet of performance has
not been related to the first one.

In this paper we will try to make first an algorithmic study of the
performance of these algorithms to establish a baseline for pool-based
volunteer
computing experiments; we will also try to make some inroads into what
kind of experiments are suitable for a volunteer computing set
up. Then we will measure the performance in a real volunteer computing
experiment. This will be done next. 

%%%%%%%%%%%%%%%%%%%%%%%%%%%%%%%%%%%%%%%%%%%%%%%%%%%%%%%%%%%%%%%%%%%%%%%%%%%%%%%
%%%%%%%%%%%%%%%%%%%%%%%%%%%%%%%%%%%%%%%%%%%%%%%%%%%%%%%%%%%%%%%%%%%%%%%%%%%%%%%

\section{Methodology} 
\label{sec:met}


\section{Experimental setup}
\label{sec:exp}


%%%%%%%%%%%%%%%%%%%%%%%%%%%%%%%%%%%%%%%%%%%%%%%%%%%%%%%%%%%%%%%%%%%%%%%%%%%%%%%
%%%%%%%%%%%%%%%%%%%%%%%%%%%%%%%%%%%%%%%%%%%%%%%%%%%%%%%%%%%%%%%%%%%%%%%%%%%%%%%

\section{Experiments and Results}
\label{sec:res}



%%%%%%%%%%%%%%%%%%%%%%%%%%%%%%%%%%%%%%%%%%%%%%%%%%%%%%%%%%%%%%%%%%%%%%%%%%%%%%%
%%%%%%%%%%%%%%%%%%%%%%%%%%%%%%%%%%%%%%%%%%%%%%%%%%%%%%%%%%%%%%%%%%%%%%%%%%%%%%%


\section{Conclusions}

\section*{Acknowledgements}

Hidden for double blind\\
But taking\\
This\\
Space


\bibliographystyle{splncs}
\bibliography{geneura-latin1,volunteer}

\end{document}
%%% Local Variables:
%%% ispell-local-dictionary: "english"
%%% hunspell-local-dictionary: "english"
%%% End:
