\documentclass[runningheads,a4paper]{llncs}

\usepackage[latin1]{inputenc}
\usepackage{amssymb}
\setcounter{tocdepth}{3}
\usepackage{graphicx}
\usepackage{subfigure}

\usepackage{url}
\newcommand{\keywords}[1]{\par\addvspace\baselineskip
\noindent\keywordname\enspace\ignorespaces#1}

\begin{document}

\mainmatter  

\title{The importance of diversity in distributed asynchronous evolutionary algorithms} 

\titlerunning{The importante of diversity}

\author{A. N. Onymouse%
\thanks{Anonymous Institute}}
%

%\author{R.~H. Garc\'ia Ortega\inst{1,2} \and P. Garc\'ia-S\'anchez\inst{1} \and J.~J. Merelo\inst{1} \and M.~G. Arenas\inst{1} \and  P.~A.~ Castillo\inst{1} \and A.~M. Mora\inst{1}}

%\authorrunning{R.~H. Garc\'ia Ortega et al.}

%\institute{Dept. of Computer Architecture and Technology, University of Granada, Spain
%\and Fundaci\'on I+D del Software Libre - Fidesol, Granada, Spain} % ¿Seguro que quieres dejar esto? - JJ

\maketitle

\begin{abstract}

Leveraging the power of all devices connected to the Internet for
distributed evolutionary algorithms requires using ubiquituous
platforms such as JavaScript on the browser. However, these platform
are heterogenous and ephemeral, which beg for a change in the canonical
evolutionary algorithm to architectures based in concepts such as a
population pool. In this paper we will test NodIO in a distributed
computing setup; first we will establish how the base pool-based algorithm itself
behaves and how it scales for different nodes; then we will test in in
a real volunteer computing environment in different circumstances to
assess what is the configuration that is able to behave the best from
the point of view of the performance. 

\keywords{Distributed evolutionary algorithms, internet computing,
  volunteer computing, pool-based evolutionary algorithms}
\end{abstract}

\section{Introduction}

The rest of the paper is organized as follows. Coming up next, we
present the state of the art in parameter setting of simulated
worlds. The next section \ref{sec:met} will present the methodology,
followed by the experimental setup in Section \ref{sec:exp}, whose results will be shown in Section \ref{sec:res}. Finally, we will
present our conclusions to finish the paper.

%%%%%%%%%%%%%%%%%%%%%%%%%%%%%%%%%%%%%%%%%%%%%%%%%%%%%%%%%%%%%%%%%%%%%%%%%%%%%%%
%%%%%%%%%%%%%%%%%%%%%%%%%%%%%%%%%%%%%%%%%%%%%%%%%%%%%%%%%%%%%%%%%%%%%%%%%%%%%%%

\section{State of the Art}
\label{sec:sota}

%%%%%%%%%%%%%%%%%%%%%%%%%%%%%%%%%%%%%%%%%%%%%%%%%%%%%%%%%%%%%%%%%%%%%%%%%%%%%%%
%%%%%%%%%%%%%%%%%%%%%%%%%%%%%%%%%%%%%%%%%%%%%%%%%%%%%%%%%%%%%%%%%%%%%%%%%%%%%%%

\section{Methodology} %Issue #12
\label{sec:met}


\section{Experimental setup}
\label{sec:exp}


%%%%%%%%%%%%%%%%%%%%%%%%%%%%%%%%%%%%%%%%%%%%%%%%%%%%%%%%%%%%%%%%%%%%%%%%%%%%%%%
%%%%%%%%%%%%%%%%%%%%%%%%%%%%%%%%%%%%%%%%%%%%%%%%%%%%%%%%%%%%%%%%%%%%%%%%%%%%%%%

\section{Experiments and Results}
\label{sec:res}



%%%%%%%%%%%%%%%%%%%%%%%%%%%%%%%%%%%%%%%%%%%%%%%%%%%%%%%%%%%%%%%%%%%%%%%%%%%%%%%
%%%%%%%%%%%%%%%%%%%%%%%%%%%%%%%%%%%%%%%%%%%%%%%%%%%%%%%%%%%%%%%%%%%%%%%%%%%%%%%


\section{Conclusions}

\section*{Acknowledgements}

Hidden for double blind\\
But taking\\
This\\
Space


\bibliographystyle{splncs}
\bibliography{geneura,volunteer}

\end{document}
%%% Local Variables:
%%% ispell-local-dictionary: "english"
%%% hunspell-local-dictionary: "english"
%%% End:
